%! Author = matt_dumont
%! Date = 20/12/23

%----------------------------------------------------------------------------------------
% Packages
%----------------------------------------------------------------------------------------
\usepackage{graphicx} % Required for including images
\usepackage[sfdefault]{carlito} % Use the Carlito font
\graphicspath{{figures/}{./}{ksl_fig/}} % Specifies where to look for included images (trailing slash required)
\usepackage[table]{xcolor}
\usepackage{array}
\usepackage{xstring}
\usepackage{lipsum}
\usepackage[utf8]{inputenc}
\usepackage[T1]{fontenc}
\usepackage[title,titletoc,toc]{appendix} % Required for the appendices section
\usepackage{titletoc} % Required for manipulating the table of contents
\usepackage{booktabs} % Required for nicer horizontal rules in tables
\usepackage{amssymb} % Required for some math elements
\usepackage{mathtools} % Required for some math elements
\usepackage[ddmmyyyy]{datetime} % Sets date format to yyyy-mm-dd
\usepackage{geometry} % Required for adjusting page dimensions and margins
\usepackage{titlesec} % Required for manipulating the section titles
\usepackage[skins, most, many, listings]{tcolorbox} % create breakaway boxes
\usepackage[backref]{hyperref} % Required for adding links and customizing them
\usepackage{enumitem, datatool} % Required for sorted list % Required for list customization
\usepackage[labelfont=bf]{caption}
\usepackage{fancyhdr} % Required for custom headers and footers
\usepackage{scrextend} % Required for indenting paragraphs
\usepackage{glossaries-extra}
\usepackage{etoolbox}
\usepackage{ksl_fig/kslfigure}
\usepackage{natbib}
\usepackage{pdflscape}
\usepackage{truncate}
\usepackage{dirtytalk}
% \usepackage[nameinlink,capitalise]{cleveref} % trialled this package, but was quite unstable! % todo multple references??
%----------------------------------------------------------------------------------------
% GLOSSARY SETUP
%----------------------------------------------------------------------------------------
\makeglossaries  %keynote if glossery isn't made then check run configurations and ensure makeglosseries is enabled as an extension
\setabbreviationstyle[acronym]{short-long}
\GlsXtrEnableEntryCounting
{acronym}% list of categories to use entry counting
{3}% trigger value (only add to glossary if count > 3)
%! Author = matt_dumont
%! Date = 4/12/23

% todo replace with general_acronym_list.tex...

\newacronym{no3n}{NO\textsubscript{3}-N}{nitrate nitrogen}
\newacronym{olw}{OLW}{Our Land and Water National Science Challenge}
\newacronym{mav}{MAV}{maximum acceptable value}
\newacronym{natbotno3}{NBL-NO\textsubscript{3}-N}{National Bottom Line for Nitrate Nitrogen}
\newacronym{ecan}{ECAN}{Environment Canterbury}
\newacronym{mfe}{MfE}{Ministry for the Environment}
\newacronym{moh}{MoH}{Ministry of Health}
\newacronym{pc1}{PC\textsubscript{1}}{Plan Change 1}
\newacronym{proj_repo}{Project Github Repo.} {Project Github Repository (https://github.com/Komanawa-Solutions-Ltd/Z22014OLW\_SelwynCaseStudy)}

\newacronym{epfm}{EPFM}{exponential piston flow model}
\newacronym{mrt}{MRT}{mean residence time}

%-----------------------------------------------------------------------------------------
% CUSTOM COLORS
%-----------------------------------------------------------------------------------------
\definecolor{ksldarkblue}{RGB}{49,83,125}
\definecolor{ksllightblue}{RGB}{64,163,219}
\definecolor{kslvlblue}{RGB}{215,236,247}
%----------------------------------------------------------------------------------------
% HYPERLINK SETUP
%----------------------------------------------------------------------------------------
\hypersetup{
    colorlinks,
    linkcolor={ksldarkblue},
    citecolor={red!80!black},
    urlcolor={blue!80!black}
}

%----------------------------------------------------------------------------------------
% define breakaway box
\newtcolorbox[auto counter,number within=section,list inside=bab]{breakawaybox}[2][]{
    enhanced,colframe=ksldarkblue,colback=kslvlblue,fonttitle=\bfseries,
    colbacktitle=ksllightblue, coltitle=black,center title,
    title=Box~\thetcbcounter: #2,#1}

\makeatletter
\newcommand\tcb@cnt@breakawayboxautorefname{Box}
\makeatother


%----------------------------------------------------------------------------------------
%	MARGINS and SPACING
%----------------------------------------------------------------------------------------
\setlength{\parindent}{0pt} % Paragraph indentation
\setlength{\parskip}{10pt} % Vertical space between paragraphs
\setcounter{tocdepth}{2} % Show entries in the table of contents down to subsections
\geometry{
    a4paper,
    total={170mm,257mm},
    left=20mm,
    top=20mm,
}
%----------------------------------------------------------------------------------------
% heading enumeration and spacing etc.
%----------------------------------------------------------------------------------------
\setcounter{secnumdepth}{4} % Number headings down to subparagraphs (level 5),
\setcounter{tocdepth}{4} % Show headings down to subparagraphs (level 5) in the table of contents

% set all section headings to ksl dark blue
\titleformat{\section}
{\normalfont\Large\bfseries\color{ksldarkblue}}{\thesection}{1em}{}

\titleformat{\subsection}
{\normalfont\large\bfseries\color{ksldarkblue}}{\thesubsection}{1em}{}

\titleformat{\subsubsection}
{\normalfont\normalsize\bfseries\color{ksldarkblue}}{\thesubsubsection}{1em}{}

\titleformat{\paragraph}
{\normalfont\normalsize\bfseries\color{ksldarkblue}}{\theparagraph}{1em}{}
\titlespacing*{\paragraph}{\parindent}{3.25ex plus 1ex minus .2ex}{.75ex plus .1ex}

\titleformat{\subparagraph}
{\normalfont\normalsize\bfseries\slshape\color{ksldarkblue}}{\thesubparagraph}{1em}{}
\titlespacing*{\subparagraph}{\parindent}{3.25ex plus 1ex minus .2ex}{.75ex plus .1ex}

%----------------------------------------------------------------------------------------
% Table formatting
%----------------------------------------------------------------------------------------
\NewTblrTheme{fancy_tab}{
    \SetTblrStyle{caption-tag}{font=\bfseries}
}
\NewTblrEnviron{ksltable}
\SetTblrOuter[ksltable]{}
\SetTblrInner[ksltable]{
    rowhead = 1,
    hlines, %horizontal lines
    row{odd} = {kslvlblue}, % alternate row colors
    row{even} = {white}, % alternate row colors
    row{1} = {font=\bfseries, ksldarkblue, fg=white}, % header row
}

\NewTblrEnviron{ksltablelong}
\SetTblrOuter[ksltablelong]{long, theme=fancy_tab}
\SetTblrInner[ksltablelong]{
    rowhead = 1,
    hlines, %horizontal lines
    vlines,
    row{odd} = {kslvlblue}, % alternate row colors
    row{even} = {white}, % alternate row colors
    row{1} = {font=\bfseries, ksldarkblue, fg=white} % header row
}

%----------------------------------------------------------------------------------------
% Figure, table, equation numbering scheme
%----------------------------------------------------------------------------------------
\numberwithin{equation}{section} % Number equations within sections (i.e. 1.1, 1.2, 2.1, 2.2 instead of 1, 2, 3, 4)
\numberwithin{figure}{section} % Number figures within sections (i.e. 1.1, 1.2, 2.1, 2.2 instead of 1, 2, 3, 4)
\numberwithin{table}{section} % Number tables within sections (i.e. 1.1, 1.2, 2.1, 2.2 instead of 1, 2, 3, 4)

%----------------------------------------------------------------------------------------
% set section/appendix autoref names
%----------------------------------------------------------------------------------------
\def\sectionautorefname{Section}
\def\subsectionautorefname{Section}
\def\subsubsectionautorefname{Section}
\def\paragraphautorefname{Section}
\def\subparagraphautorefname{Section}
\def\appendixautorefname{Appendix}

%----------------------------------------------------------------------------------------
% appendix autoref patch from https://tex.stackexchange.com/questions/149807/autoref-subsections-in-appendix
%----------------------------------------------------------------------------------------

\makeatletter
\patchcmd{\hyper@makecurrent}{%
    \ifx\Hy@param\Hy@chapterstring
    \let\Hy@param\Hy@chapapp
    \fi
}{%
    \iftoggle{inappendix}{%true-branch
    % list the names of all sectioning counters here
        \@checkappendixparam{chapter}%
        \@checkappendixparam{section}%
        \@checkappendixparam{subsection}%
        \@checkappendixparam{subsubsection}%
        \@checkappendixparam{paragraph}%
        \@checkappendixparam{subparagraph}%
    }{}%
}{}{\errmessage{failed to patch}}

\newcommand*{\@checkappendixparam}[1]{%
    \def\@checkappendixparamtmp{#1}%
    \ifx\Hy@param\@checkappendixparamtmp
    \let\Hy@param\Hy@appendixstring
    \fi
}
\makeatletter

\newtoggle{inappendix}
\togglefalse{inappendix}

\apptocmd{\appendix}{\toggletrue{inappendix}}{}{\errmessage{failed to patch}}
\apptocmd{\subappendices}{\toggletrue{inappendix}}{}{\errmessage{failed to patch}}
% end appendix autoref patch


%----------------------------------------------------------------------------------------
% define inline equation
%----------------------------------------------------------------------------------------
\makeatletter
\newcommand*{\inlineequation}[2][]{%
    \begingroup
    % Put \refstepcounter at the beginning, because
    % package `hyperref' sets the anchor here.
    \refstepcounter{equation}%
    \ifx\\#1\\%
    \else
    \label{#1}%
    \fi
    % prevent line breaks inside equation
    \relpenalty=10000 %
    \binoppenalty=10000 %
    \ensuremath{%
    % \displaystyle % larger fractions, ...
        #2%
    }%
    ~\@eqnnum
    \endgroup
}
\makeatother

%----------------------------------------------------------------------------------------
% define get year from date
\def\getYear#1{\StrRight{#1}{4}}