%! Author = matt_dumont
%! Date = 14/12/23

\section[Conclusions]{Conclusions} \label{sec:conclusions}

We conclude that:
\begin{itemize}
    \item The lack of observed reductions in \gls{no3n} across the Selwyn Waihora catchment is not inconsistent with full implementation of \gls{pc1} reductions.
    \item With a full 20\% reduction in \gls{no3n} loads in the source area only 66\% of the groundwater wells assessed are likely to have decreasing \gls{no3n} concentrations from 2017 concentrations. The reductions in the other 33\% will simply achieve a lower steady state concentration.
    \item If we assume that dairy farms only make up 50\% of every source area then we would only expect decreasing concentrations from 2017 levels in 52\% of monitoring wells.
    \item With the current monitoring network and the current quarterly sampling, reductions should be detectable in c. 50\% of the monitoring wells by 2062, increasing sampling frequencies to monthly or weekly would allow detection of the reductions in c. 50\% of the monitoring wells by 2042 or 2037, respectively. Recall that these estimates exclude the 33\% of wells which will simply achieve a lower steady state concentration.
    \item The lack of \gls{mrt} assessments in the surface water features precludes a robust assessment of the detection power of the surface water features. However, we have produced estimates of the detection power assuming a range of \gls{mrt} values.
    \item Assessments of \gls{mrt} in surface water features can also help to constrain the likely maximum \gls{no3n} concentration in a stream that will arise from current and past land use.
    \item The current groundwater monitoring frequency likely provides as significant a barrier to the detection of reductions as the intrinsic lag of the groundwater system.
    \item The current monitoring network is not well suited to the detection of reductions in \gls{no3n} concentrations in the Selwyn Waihora catchment.
\end{itemize}

\section[Recommendations]{Recommendations} \label{sec:recommendations}

Based on the work presented here we recommend the following to improve the likelihood of detecting reductions in \gls{no3n} concentrations in the Selwyn Waihora catchment:
\begin{itemize}
    \item Additional \gls{mrt} assessments should be undertaken in the Selwyn Waihora catchment to better constrain the detection power and the likely maximum \gls{no3n} concentration in streams and groundwater wells that will arise from current and past land use.
    \item \gls{mrt} assessments should be prioritised for the spring fed streams as these are the most sensitive receptors and act as integrators of the catchment.
    \item Once the \gls{mrt} assessments are complete, the detection power of the surface water features and any groundwater monitoring locations with new \gls{mrt} values should be re-assessed.
    \item Increasing the sampling frequency is essential to improve the detection power of the monitoring network. We recommend that targeted sites undergo higher frequency monitoring in order to meet detection timeline requirements. Once higher frequency data is available the detection power of these sites should be re-evaluated to ensure that the novel data does not affect the detection power.
    \item Additional monitoring wells which target young waters may be useful; however there is a risk that new young wells may already be at or near steady state which would confound the detection of \gls{pc1} reductions. Minimally the network should be reviewed and likely expanded prior to any further mandated reduction.
    \item More sophisticated statistical analysis could and should be used to extract additional information from the existing data
\end{itemize}