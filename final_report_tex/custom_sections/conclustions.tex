%! Author = matt_dumont
%! Date = 14/12/23

\section[Conclusions]{Conclusions} \label{sec:conclusions}

We conclude that:
\begin{itemize}
    \item The lack of observed reductions in \gls{no3n} across the Selwyn Waihora catchment is not inconsistent with full implementation of \gls{pc1} reductions. Monitoring results to date provide therefore no information on whether \gls{pc1} has reduced nitrate concentrations in the catchment.
    \item With a full 20\% reduction in \gls{no3n} loads in the source area only 66\% of the groundwater wells assessed are likely to show decreasing \gls{no3n} concentrations at any point in the future. The reductions in the other 33\% will simply achieve a lower steady state concentration than would have occurred in the absence of nitrate loss reductions.
    \item If we assume an average reduction of 10\% then we would only expect decreasing \gls{no3n} concentrations in 52\% of monitoring wells.
    \item With the current monitoring network, the current quarterly sampling regime, and an assumed full 20\% reduction; only 15 of the monitoring wells are likely to show a decreasing \gls{no3n} trend by 2062, increasing sampling frequencies to monthly or weekly would allow detection of the reductions in these 15 wells by 2042 or 2037, respectively.
    \item The lack of \glsfirst{mrt} assessments in the surface water bodies precludes a robust assessment of the detection power of these sites. We have therefore produced estimates of the detection power for these sites under an assumed range of \gls{mrt} values. The results highlight the importance of obtaining water age data for surface watercourses to understand whether actions undertaken to reduce nitrate concentrations have been successful.
    \item Assessments of \gls{mrt} in surface water features would also help to constrain the likely maximum future \gls{no3n} concentration when the full effects of past land use reach the stream and the steady state concentration when water quality equilibrates with current nitrate losses from the soil profile.
    \item Statistical power limitations associated with the current groundwater monitoring frequency is likely to constrain detection of nitrate loss reductions to the same degree as hydrological lags. Both factors must be considered together when drawing conclusions from nitrate monitoring results. Failing to do this would equate to high risk of statistical error. If the monitoring program detects a trend when none is present (Type I error), fails to detect a real trend (Type II error), or estimates a trend that is opposite to the one present (Type III error), any management decisions based on the monitoring results could undermine rather than support the management objectives.
    \item The current monitoring network is not well suited to the detection of reductions in \gls{no3n} concentrations in the Selwyn Waihora catchment; however the network serves multiple purposes some of which are counter to high detection powers (e.g., characterising the state of the deep aquifer system). Note we have not assessed the spatial representativeness of the monitoring network in this analysis as it was beyond the scope of this project.
\end{itemize}

\section[Recommendations]{Recommendations} \label{sec:recommendations} % todo stopped review here!, % todo start here

Based on the work presented here we recommend the following to improve the likelihood of detecting reductions in \gls{no3n} concentrations in the Selwyn Waihora catchment:
\begin{itemize}
    \item The monitoring design framework presented in \citet{olw_guidance} should be applied to the Selwyn Waihora catchment.
    \item Water age sampling and \gls{mrt} assessments should be undertaken for a prioritised set of monitoring sites to better constrain the detection power and the likely maximum \gls{no3n} concentration in streams and groundwater wells that will arise from current and past land use.
    \item Spring fed streams should be assigned a high monitoring priority due to their sensitivity and role as integrators of the catchment water quality over a broader area than individual monitoring wells.
    \item \gls{mrt} assessments should be highly prioritised for the spring fed streams.
    \item Once the \gls{mrt} assessments are complete, the detection power of the surface water features and any groundwater monitoring locations with new \gls{mrt} values should be re-assessed.
    \item Increasing the sampling frequency is essential to improve the detection power of the monitoring network. We recommend that targeted sites undergo higher frequency monitoring in order to meet detection timeline requirements. Once higher frequency data is available the detection power of these sites should be re-evaluated to ensure that the novel data does not change the detection power.
    \item Additional monitoring wells which target young waters may be useful; however there is a risk that new young wells may already be at or near steady state which would confound the detection of \gls{pc1} reductions. The network should be reviewed and bespoke monitoring locations should be developed prior to any further mandated reduction in nitrate losses.
    \item More sophisticated statistical analysis could and should be used to extract additional information from the existing data.
\end{itemize}