%! Author = matt_dumont
%! Date = 14/12/23

\section[Introduction]{Introduction} \label{sec:intro}

\gls{no3n} is a contaminant of significant concern both worldwide and in Aotearoa / New Zealand.
At concentrations > 0.8 mg/l, \gls{no3n} can stimulate the growth of periphyton and phytoplankton\citep{mcdowell_global_2020}; concentrations greater than 2.4 mg/l can cause toxicological effects to in stream fauna
\citep{camargo_nitrate_2005, horak_assessing_2019,wagenhoff_identifying_2017};
finally concentrations above the \gls{mav} (>11.3 mg/l) can cause health impacts in humans\citep{rahman_anthropogenic_2021}.

The Selwyn Waihora zone is a large catchment in the Canterbury plains south of Christchurch New Zealand. The catchment stretches from the foothills of the Southern Alps to the coast. There is significant agricultural activity, as well as many high value ecosystems including Te Waihora / Lake Ellesmere, Te Roto o Wairewa / Lake Forsyth, and the Kaitorete Spit. The Selwyn Waihora zone is also home to many significant cultural sites for Ngāi Tahu, the local iwi. \gls{no3n} concentrations in the Selwyn Waihora zone are often elevated with many surface water sites exceeding the \gls{natbotno3} of 2.4 mg/l\citep{noauthor_national_2020}. \gls{ecan} made \gls{pc1} operative from 1st February 2016 to begin to address the issue of elevated \gls{no3n} concentrations in the Selwyn Waihora zone. \gls{pc1} was designed to achieve this by reducing the amount of nitrogen leaching from agricultural land in the Selwyn Waihora zone by requiring high \gls{no3n} leaching land uses required to implement \gls{no3n} leaching reductions of at least 20\%. Other land uses were also required to make lower reductions. \gls{no3n} reductions were expected to begin implementation in 2017 and should have been fully implemented by 2022.

Many landowners have stated they have already fully implemented the required reductions;
however, \gls{no3n} concentrations at most sites in the Selwyn Waihora zone have not yet shown any significant reductions\citep{scottpc}.
This discrepancy could be caused by a number of factors including:
\begin{itemize}
    \item Lag between on farm mitigations and in stream \gls{no3n} concentrations.
    \item Insufficient precision in the \gls{no3n} monitoring network.
    \item Historical increases in \gls{no3n} concentrations that have yet to reach steady state with the monitoring network.
    \item Less effective than expected on farm mitigations.
    \item Poor or incomplete implementation of on farm mitigations.
    \item Other factors such as climatic variations impacting groundwater recharge.
\end{itemize}
Given the political, ecological, and economic sensitivity of the catchment \gls{ecan} commissioned this study to better understand: 1) when the implemented \gls{no3n} concentration reductions should be observable in the current monitoring network and 2) what barriers may exist in detecting \gls{no3n} reductions in the Selwyn Waihora zone.
