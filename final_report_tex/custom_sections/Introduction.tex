%! Author = matt_dumont
%! Date = 14/12/23

\section[Introduction]{Introduction} \label{sec:intro}

\gls{no3n} is a contaminant of significant concern both worldwide and in Aotearoa / New Zealand.
At concentrations > 0.8 mg/l, \gls{no3n} can stimulate the growth of periphyton and phytoplankton\citep{mcdowell_global_2020}; concentrations greater than 2.4 mg/l can cause toxicological effects to in stream fauna
\citep{camargo_nitrate_2005, horak_assessing_2019,wagenhoff_identifying_2017};
finally concentrations above the \gls{mav} (>11.3 mg/l) can cause human health impacts\citep{rahman_anthropogenic_2021}.

\begin{wrapfigure}{R}{0.6\textwidth}
    \begin{breakawaybox}[label={box:no3_red}]{\gls{no3n} reduction required by \gls{pc1}}
        \begin{itemize}
            \item 30\% for dairy
            \item 2\% for dryland sheep, beef or deer
            \item 22\% for dairy support
            \item 7\% for arable
            \item 20\% for pigs
            \item 5\% for fruit, viticulture or vegetables
            \item 5\% for irrigated sheep, beef or deer
        \end{itemize}
    \end{breakawaybox}
\end{wrapfigure}

The Selwyn Waihora zone is a large catchment in the Canterbury plains south of Christchurch.
The catchment stretches from the foothills of the Southern Alps to the coast.
There is significant agricultural activity, as well as many high value ecosystems including Te Waihora / Lake Ellesmere and the Kaitorete Spit.
The Selwyn Waihora zone is also home to many significant cultural sites for Ngāi Tahu, the local iwi.
\gls{no3n} concentrations in the Selwyn Waihora zone are elevated with many surface water sites exceeding the \gls{natbotno3} of 2.4 mg/l\citep{noauthor_national_2020}.
\gls{pc1} of the Canterbury Land and Water Regional Plan, operative from 1st February 2016, includes provisions to reduce nitrate leaching concentrations in the Selwyn Waihora; however there was acknowledgment \gls{no3n} concentrations could initially continue to increase after implementing \gls{pc1} reduction.
In addition, \gls{pc1} predated the National Policy Statement for Freshwater Management 2020 and therefore the \gls{natbotno3} of 2.4 mg/l was not a requirement of \gls{pc1}.
\gls{pc1} \gls{no3n} reductions ranged from 2-30\% \autoref{box:no3_red}.
Implementation of these \gls{no3n} reductions were expected to begin in 2017 and should have been fully implemented by 2022.


Environment Canterbury maintains a network of monitoring wells and surface water sites to track \gls{no3n} concentrations in the Selwyn Waihora zone as is required by the The Resource Management Act 1991.
The monitoring programmes are reviewed periodically with the most recent review of the groundwater quality and water level network being in 2022.
The state purpose of the groundwater monitoring network is to:
\begin{itemize}
    \item Monitor long-term groundwater state and trends.
    \item Improve scientific understanding of Canterbury groundwater systems and help Environment Canterbury manage groundwater in the region.
    \item Assess progress against freshwater outcomes.
    \item Inform the effectiveness of regional policies and plans \citep{ecan_monitor_review}.
\end{itemize}

Although many landowners have stated they have already fully implemented the required reductions\citep{scottpc}, \gls{no3n} concentrations at most sites in the Selwyn Waihora zone have not yet shown any significant reductions\citep{scottpc}.
This discrepancy could be caused by a number of factors including:
\begin{itemize}
    \item Lag
    \begin{itemize}
        \item \gls{no3n} concentrations have not yet reached steady state with the monitoring network.
        \item Historical increases in \gls{no3n} concentrations that have yet to reach steady state with the monitoring network.
        \item \gls{no3n} stored in the unsaturated (vadose) zone.
    \end{itemize}
    \item Insufficient precision in the \gls{no3n} monitoring network (a lack of detection power).
    \item Nitrate loss mitigations are less effective than expected.
    \item The difference in \gls{no3n} load and \gls{no3n} leachate concentration. \gls{pc1} reductions are defined as percent reductions in \gls{no3n} load. Changes in coincident land surface recharge changes (e.g., more efficient irrigation) can yield increasing or decreasing \gls{no3n} leachate concentration.
    \item Poor or incomplete implementation of on-farm mitigations.
    \item Other factors such as climatic variations and boundary condition changes (e.g., changes in losses from leaky water races) impacting groundwater recharge.
\end{itemize}

The purpose of this study is to better understand: 1) when the implemented reductions should be observable in the current monitoring network and 2) obstacles to detecting \gls{no3n} reductions in the Selwyn Waihora zone.
We focus on the potential statistical errors that can arise from the monitoring network design.
The monitoring network detecting a trend when none is present (Type I error), failing to detect a real trend (Type II error), or estimating a trend that is opposite to the one present (Type III error); could affect any management decisions based on the monitoring results could undermine rather than support the management objectives.
