%! Author = matt_dumont
%! Date = 14/12/23

\section[Introduction]{Introduction} \label{sec:intro}

\gls{no3n} is a contaminant of significant concern both worldwide and in Aotearoa / New Zealand.
At concentrations > 0.8 mg/l, \gls{no3n} can stimulate the growth of periphyton and phytoplankton\citep{mcdowell_global_2020}; concentrations greater than 2.4 mg/l can cause toxicological effects to in stream fauna
\citep{camargo_nitrate_2005, horak_assessing_2019,wagenhoff_identifying_2017};
finally concentrations above the \gls{mav} (>11.3 mg/l) can cause human health impacts\citep{rahman_anthropogenic_2021}.

The Selwyn Waihora zone is a large catchment in the Canterbury plains south of Christchurch.
The catchment stretches from the foothills of the Southern Alps to the coast.
There is significant agricultural activity, as well as many high value ecosystems including Te Waihora / Lake Ellesmere, Te Roto o Wairewa / Lake Forsyth, and the Kaitorete Spit.
The Selwyn Waihora zone is also home to many significant cultural sites for Ngāi Tahu, the local iwi.
\gls{no3n} concentrations in the Selwyn Waihora zone are elevated with many surface water sites exceeding the \gls{natbotno3} of 2.4 mg/l\citep{noauthor_national_2020}.
\gls{pc1} of the Canterbury Land and Water Regional Plan, operative from 1st February 2016, includes provisions to reduce nitrate concentrations in the Selwyn Waihora zone.
Intensive land users are required to implement \gls{no3n} leaching reductions of at least 20\%.
Other land uses were also required to make lower reductions.
Implementation of these \gls{no3n} reductions were expected to begin in 2017 and should have been fully implemented by 2022.

Although many landowners have stated they have already fully implemented the required reductions\citep{scottpc}, \gls{no3n} concentrations at most sites in the Selwyn Waihora zone have not yet shown any significant reductions\citep{scottpc}.
This discrepancy could be caused by a number of factors including:
\begin{itemize}
    \item Lag between on farm mitigations and in stream \gls{no3n} concentrations.
    \item Insufficient precision in the \gls{no3n} monitoring network.
    \item Historical increases in \gls{no3n} concentrations that have yet to reach steady state with the monitoring network.
    \item Nitrate loss mitigations are less effective than expected.
    \item Poor or incomplete implementation of on-farm mitigations.
    \item Other factors such as climatic variations impacting groundwater recharge.
\end{itemize}
The purpose of this study is to better understand: 1) when the implemented \gls{no3n} concentration reductions should be observable in the current monitoring network and 2) obstacles to detecting \gls{no3n} reductions in the Selwyn Waihora zone.
We focus on the potential statistical errors that can arise from the monitoring network design.
The monitoring network detecting a trend when none is present (Type I error), failing to detect a real trend (Type II error), or estimating a trend that is opposite to the one present (Type III error); could affect any management decisions based on the monitoring results could undermine rather than support the management objectives.
