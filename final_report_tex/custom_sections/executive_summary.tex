%! Author = matt_dumont
%! Date = 4/12/23

\phantomsection % add to toc (make the link go to the right location)
\addcontentsline{toc}{section}{Executive Summary} % add to toc
\section*{Executive Summary} \label{exsum} % the asterisk prevents a section number from being assigned

\gls{no3n} concentrations are elevated in the Selwyn Waihora catchment and \gls{ecan} has implemented \gls{pc1} to address these high \gls{no3n} concentrations in streams and groundwater. \gls{pc1} became operational in 2016 and \gls{no3n} reductions should have been fully implemented as of 2022. Despite this very few monitored sites in the Selwyn Waihora catchment show decreasing \gls{no3n} concentrations. Here we have undertaken detection power analysis of 56 sites (46 groundwater wells and 10 surface water features) to identify when \gls{pc1} reductions could be detected in the current monitoring network and to identify any barriers to their detection.

Our analysis suggests that the current monitoring network is not well suited to detecting reductions in \gls{no3n} at the scale that was implemented in \gls{pc1} ($\leq$20\%). Only 52-66\% of groundwater monitoring sites should expect to ever have reducing \gls{no3n} concentrations (Assuming average source zone reductions of 10-20\%). The remaining 33-48\% of sites will simply achieve a lower steady state concentration than if \gls{pc1} had not been implemented. There is significant uncertainty in these estimates due to the lack of \gls{mrt} assessments in c. 60\% of the groundwater monitoring locations we investigated. However, is our \gls{mrt} estimates are correct then \gls{pc1} reductions should be detectable in c. 50\% of the monitoring wells by 2062, increasing sampling frequencies to monthly or weekly would allow detection of the reductions in c. 50\% of the monitoring wells by 2042 or 2037, respectively. These estimates exclude the 33-48\% of wells which will simply achieve a lower steady state concentration.

The lack of \gls{mrt} assessments in the surface water features precludes a robust assessment of the detection power; however, we have produced detection power estimates by assuming a range of \gls{mrt} values. The surface water features act as an integrator of the catchment, so better understanding their detection power would significantly alleviate the challenges of developing a spatially representative network. In addition, assessments of \gls{mrt} in surface water features can help to constrain the likely maximum \gls{no3n} concentration in a stream that will arise from current and past land use. As an example our, notably simplistic, analysis of Hart's creek suggest that the maximum \gls{no3n} concentration could as low as 7.6 mg/l if assuming a \gls{mrt} of 5 years, but could be as high as 12.0 mg/l if we assume a \gls{mrt} of 30 years. We would like to highlight that these figures are not predictions; which are possible, but require more sophisticated statistical analysis; but rather are intended to illustrate the uncertainty in the maximum \gls{no3n} concentration that could arise from current and past land use in the absense of \gls{mrt} constraints.

We conclude that the current groundwater monitoring network is likely insufficient to detect \gls{pc1} reductions in \gls{no3n} concentrations in the Selwyn Waihora catchment in a timely manner. The current groundwater monitoring frequency likely provides as significant a barrier to the detection of reductions as the intrinsic lag of the groundwater system. In order to improve the likelihood of detecting reductions in \gls{no3n} concentrations in the Selwyn Waihora catchment we recommend:
\begin{itemize}
    \item Undertaking additional \gls{mrt} assessments, particularly in the surface water features.
    \item Increasing the sampling frequency of targeted sites to achieve the required reduction detection timelines.
    \item Reviewing the monitoring network and likely expanding it prior to any further mandated reductions.
    \item And finally, undertaking more sophisticated statistical analysis to extract additional information from the existing data.
\end{itemize}
